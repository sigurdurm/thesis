%% ----------------------------------------------------------------
%% Thesis.tex -- MAIN FILE (the one that you compile with LaTeX)
%% ---------------------------------------------------------------- 

% Set up the document
\documentclass[a4paper, 11pt, oneside]{Thesis}  % Use the "Thesis" style, based on the ECS Thesis style by Steve Gunn
\graphicspath{{Figures/}}  % Location of the graphics files (set up for graphics to be in PDF format)

% Include any extra LaTeX packages required
\usepackage[square, numbers, comma, sort&compress]{natbib}  % Use the "Natbib" style for the references in the Bibliography
\usepackage{verbatim}  % Needed for the "comment" environment to make LaTeX comments
\usepackage{vector}  % Allows "\bvec{}" and "\buvec{}" for "blackboard" style bold vectors in maths
\hypersetup{urlcolor=blue, colorlinks=true}  % Colours hyperlinks in blue, but this can be distracting if there are many links.

\usepackage{lmodern} %Do get the right font encoding

%\setlength{\parindent}{0.8cm}
%\setlength{\parskip}{0pt}
%% ----------------------------------------------------------------
\begin{document}
\frontmatter	  % Begin Roman style (i, ii, iii, iv...) page numbering

% Set up the Title Page
\title  {Clustering Player Behaviors in Data Streams with K-Means in Map-Reduce}
\authors  {\texorpdfstring
            {\href{sigurdurm@gmail.com}{Sigurdur Karl Magnusson}}
            {Sigurdur Karl Magnusson}
            }
\addresses  {\groupname\\\deptname\\\univname}  % Do not change this here, instead these must be set in the "Thesis.cls" file, please look through it instead
\date       {\today}
\subject    {}
\keywords   {}

\maketitle
%% ----------------------------------------------------------------

\setstretch{1.3}  % It is better to have smaller font and larger line spacing than the other way round

% Define the page headers using the FancyHdr package and set up for one-sided printing
\fancyhead{}  % Clears all page headers and footers
\rhead{\thepage}  % Sets the right side header to show the page number
\lhead{}  % Clears the left side page header

\pagestyle{fancy}  % Finally, use the "fancy" page style to implement the FancyHdr headers


%% ----------------------------------------------------------------
% The "Funny Quote Page"
\pagestyle{empty}  % No headers or footers for the following pages

\null\vfill
% Now comes the "Funny Quote", written in italics
\textit{``Any intelligent fool can make things bigger, more complex, and more violent. It takes a touch of genius -- and a lot of courage -- to move in the opposite direction.''}

\begin{flushright}
Albert Einstein
\end{flushright}

\vfill\vfill\vfill\vfill\vfill\vfill\null
\clearpage  % Funny Quote page ended, start a new page
%% ----------------------------------------------------------------

% The Abstract Page
\chapter*{Abstract}
\addcontentsline{toc}{chapter}{Abstract}
\thispagestyle{plain}

The abstract is a short summary of the thesis. It announces in
a brief and concise way the scientific goals, methods, and most important
results. The chapter ``conclusions'' is not equivalent to the abstract!
Nevertheless, the abstract may contain concluding remarks. The abstract
should not be discursive. Hence, it cannot summarize all aspects of the thesis
in very detail. Nothing should appear in an abstract that is not also
covered in the body of the thesis itself. Hence, the abstract should be the
last part of the thesis to be compiled by the author.

A good abstract has the following properties: \emph{Comprehensive:} All major
parts of the main text must also appear in the abstract. \emph{Precise:}
Results, interpretations, and opinions must not differ from the ones in the main
text. Avoid even subtle shifts in emphasis. \emph{Objective:} It may contain
evaluative components, but it must not seem judgemental, even if the thesis
topic raises controversial issues. \emph{Concise:} It should only contain the
most important results. It should not exceed 300--500 words or about one page.
\emph{Intelligible:} It should only contain widely-used terms. It should
not contain equations and citations. Try to avoid symbols and acronyms (or at
least explain them). \emph{Informative:} The reader should be able to quickly
evaluate, whether or not the thesis is relevant for his/her work.

An Example: The objective was to determine whether \dots (\emph{question/goal}).
For this purpose, \dots was \dots (\emph{methodology}). It was found that \dots
(\emph{results}). The results demonstrate that \dots (\emph{answer}).

\clearpage  % Abstract ended, start a new page
%% ----------------------------------------------------------------

\setstretch{1.3}  % Reset the line-spacing to 1.3 for body text (if it has changed)

% The Acknowledgements page, for thanking everyone
\acknowledgements{
\addtocontents{toc}{\vspace{1em}}  % Add a gap in the Contents, for aesthetics

The acknowledgments and the people to thank go here, don't forget to include your project advisor\ldots

}
\clearpage  % End of the Acknowledgements
%% ----------------------------------------------------------------

\pagestyle{fancy}  %The page style headers have been "empty" all this time, now use the "fancy" headers as defined before to bring them back


%% ----------------------------------------------------------------
\lhead{\emph{Contents}}  % Set the left side page header to "Contents"
\tableofcontents  % Write out the Table of Contents

%% ----------------------------------------------------------------
\lhead{\emph{List of Figures}}  % Set the left side page header to "List if Figures"
\listoffigures  % Write out the List of Figures

%% ----------------------------------------------------------------
\lhead{\emph{List of Tables}}  % Set the left side page header to "List of Tables"
\listoftables  % Write out the List of Tables

%% ----------------------------------------------------------------
\setstretch{1.5}  % Set the line spacing to 1.5, this makes the following tables easier to read
\clearpage  % Start a new page
\lhead{\emph{Abbreviations}}  % Set the left side page header to "Abbreviations"
\listofsymbols{ll}  % Include a list of Abbreviations (a table of two columns)
{
% \textbf{Acronym} & \textbf{W}hat (it) \textbf{S}tands \textbf{F}or \\
\textbf{LAH} & \textbf{L}ist \textbf{A}bbreviations \textbf{H}ere \\

}

%% ----------------------------------------------------------------
\mainmatter	  % Begin normal, numeric (1,2,3...) page numbering
\pagestyle{fancy}  % Return the page headers back to the "fancy" style

% Include the chapters of the thesis, as separate files
% Just uncomment the lines as you write the chapters

% Chapter 1

\chapter{Chapter Title Here} % Write in your own chapter title
\label{Chapter1}
\lhead{Chapter 1. \emph{Introduction}} % Write in your own chapter title to set the page header

\section{Motivation}
We live in a time of data, where data is growing at an amazing rate. Data is everywhere around us and companies are generating more and more data.
Storing data is cheaper than ever and many smaller and mid-sized companies are buying storage subscriptions at other bigger companies that offer "Storage as a Service" (SaaS). A SaaS company offers e.g. data reliability and durability by storing critical data in multiple facilities and on multiple devices. 
Having all this data does not give a meaning by itself, it needs to be analysed and interpret to give a business value.
\null\vfill
% Now comes the "Funny Quote", written in italics
\textit{``Information is the oil of the 21st century, and analytics is the combustion engine.''}

\begin{flushright}
Peter Sondergaard, senior VP at Gartner
\end{flushright}


The need to analyse all this data have caused many new companies to rise up that are specialised in analysing large quantities of data, extracting knowledge and converting it to a business value. Many of those companies offer "Analytics as a Service" (AaaS) that offer an analytical software to discover trends and unknown patterns in the data.

\section{Problem Statement}
GameAnalytics.com is a cloud hosted service for collecting, analyzing and reporting game metrics. Working with large quantities of game metric data that needs to be analysed and processed efficiently. 

A valuable application for GA is to design and implement a scalable streaming version of a clustering algorithm. That can read a large data in mini-batches into memory and cluster the data incrementally. 

The goal would be to incrementally find clusters efficiently in a streaming data, showing predominant characteristics of human behavior, e.g. the hardcore players, casual players, people who didn't understand the game, etc.

\section{Contributions}
\begin{itemize}
\item Clustering Player Behaviours using Map-Reduce framework
\item Software and knowledge to GameAnalytics for further development
\end{itemize}

\section{Project Outline}

 % Introduction

% Chapter 2

\chapter{Background Theory} % Write in your own chapter title
\label{Chapter2}
\lhead{Chapter 2. \emph{Background Theory}} % Write in your own chapter title to set the page header


\section{Player Behavior Profiles}
\textit{TODO Describe player behavior}

\lipsum[1-2]

\subsection{Game Metric}

\textit{TODO Describe User Telemetry and Game Metric. Features and behavioral variables}

\lipsum[2-3]

\section{Clustering}
The goal of clustering is to categories or groups similar objects together into so called clusters (hidden data structures) while different objects belong to other clusters. A cluster are set of objects that are similar to each other, while objects in different clusters are dissimilar to each other. Identifying descriptive features of an object one can compare these features to a known object based on their similarity or dissimilarity based on some criteria. Cluster analysis can be achieved by various algorithms and is a common technique in statistical data analysis that is used in many fields, e.g. machine learning, pattern recognition, image analysis and bioinformatics. In this thesis we focus on a popular clustering algorithm called k-means that is a centroid based clustering algorithm where a cluster is represented by a central vector called centroid.

\subsection{K-means clustering method}
Many clustering methods exists but one of the most popular ones is called \textit{k-means} \citep{FORGYE.W.:1965,MacQueen:1967KMeans}, also known as the Lloyd algorithm \citep{Lloyd:1982} which was further generalized for vector quantization \citep{Linde:1980VQ}. K-means seeks to group similar data points into \textit{k} partitions or hyperspherical clusters giving insights into the general distributions in the dataset. A cluster is represented by a centroid that characterize the geometric center of the cluster that is calculated as the mean of all data instances belonging to that cluster. The objective function in k-means is to minimizes the squared error between a cluster's centroid (mean) and its assigned points and over all set of clusters minimizing the Sum of Squared Error (SSE). Let a set of data points $x_i \in \Re^d, i=1,...,N$, where each point $x$ is a real number $d$-dimensional vector and we want to partition them into \textit{K} clusters $C=\{c_1,...,c_K\} \in \Re^d$, then the objective function is defined as

\begin{center}
$SSE =\displaystyle \sum_{k=1}^{K}\displaystyle \sum_{x_i \in c_k}\left \| x_i-\mu_k \right \|^2$ 
\end{center}


where $\mu_k$ is the mean vector for the cluster centroid $k$ and $\left \| x_i-\mu_k \right \|$ is a Euclidean distance measure between two points the data vector $x_i$ and the mean vector $\mu_k$, calculated respectively

\begin{center}
$\mu_k = \dfrac{1}{n_k}\displaystyle \sum_{x_i \in c_k}x_i,$ 
\end{center}

where $c_k$ is a cluster number $k$ and its $n_k$ data points $i=1,...,n_k$. Euclidean distance function is defined as

\begin{center}
$\left \| x_i-y_i \right \| = \sqrt{\displaystyle \sum_{i=1}^{d}(x_i-y_i)^2} $
\end{center}

When running the k-means algorithm it starts by initializing $K$ centroids by choosing random data points from the dataset or according some heuristic procedure. In each iteration of the algorithm it assigns data points to it's nearest centroid by calculating the minimum Euclidean distance to the $K$ centroids for each instance. After assigning all the data points to clusters the centroids are updated so they represent the mean value of all the points in the corresponding cluster. The algorithm stops the iteration when the centroids do not change from the previous iteration or the error (SSE) is below some specific threshold. Also possible to manually define a maximum number of iterations to be run. 


\textit{TODO INSERT PICTURE, showing some simple iterations...}


One of the weaknesses of k-means is that it is sensitive of the initial selection of the centroids which can lead to local optimum, that is the algorithm converges and fails to find the global optimum. One solution is to run the algorithm $n$ times and pick the initialization that gave the lowest SSE result. Another weakness of k-means is that a user has to predefine the number of centers k-means need to cluster, this is most often not known in advance. Many methods exist and most popular one is to run the algorithm with by increasing the $k$ number of clusters to some $K$ and pick a good $k$ candidate where the \textit{``elbow''} starts in the curve in a Scree plot \citep{Han:2006DM}. Noise in data and outliers can dramatically increase the squared error and centroids shifting from data distribution in question towards outliers far away, thus representing skewed distributions. Solutions involve removing these noise in preprocessing or normalize the data with the zero mean normalization \citep{Xu:2005Clustering, Han:2006DM}.

%Example about local optimum%
%For example imagine if the global optimum is three cluster centroids each representing three separate data distributions then the algorithm fails to find the global optimum %if it initializes two centroids in one distribution and the last centroid in between the other two distributions. The algorithm stops and converges to local optimum.

The solution to the optimal partition can be found by checking all possibilities using a brute force method but that is a $NP$-hard problem \cite{Aloise:2009KmeansNPHard} and cannot be solved in a reasonable time. The k-means algorithm is a heuristic approach for the clustering problem with running time of the algorithm $O(NKdT)$ where $N$ is number data examples in $d$-dimensional space and $T$ is the number of iterations. Usually $K,d$ and $T$ is much less than $N$ meaning that k-means is good for clustering large-scale data because of approximately linear time complexity.

The above implementation of k-means is called the \textit{batch} k-means, where the centroids are updated after all the data points have been assigned. The \textit{online} (incremental) mode of the algorithm processes each data point sequentially. For each data point $x$ the nearest cluster centroid $c_{min}$ is calculated and that centroid is updated right away, defined as 

\begin{center}
$c_{min}(old) = \underset{k}{\operatorname{argmin}} \|x-c_k\|$

$c_{min}(new) = c_{min}(old)+\eta(x-c_{min}(old))$
\end{center}

where the cluster centroid $c_{min}$ is updated towards the data point $x$ using the learning rate $\eta$ which determines the adaptation speed to each data point. The online approach is however highly dependent on the order of which the data points are processed. A variant of this method is used when clustering an endless stream of data where data points arrive one at a time or in chunks.


\subsection{Player Behaviors}
\textit{TODO Describe clustering player behaviors with focus using k-means}

K-means algorithm have been shown to be very useful in behavioral analysis to give good insights in the general behaviors found in a game [CITE].

\lipsum[6]


\section{MapReduce and Large-Scale Data}
MapReduce is a programming model introduced by Google in 2004 \citep{Dean:2004}, built on the divide-and-conquer paradigm, dividing a large-scale data into smaller chunks and process them in parallel. MapReduce enables fault-tolerant distributed computing on large-scale datasets and is a new way to interact with \textit{Big Data} where as old techniques are more complicated, costly and time consuming \citep{Dean:2004}. Google also introduced along with MapReduce a powerful distributed file system called \textit{Google File System} (GFS) that could hold massive amount of data. This led to a new open source software framework called Hadoop \citep{bialecki2005hadoop}, written in Java and is now maintained by Apache Foundation, a end-to-end solution for organizations that want to apply MapReduce. There are many Hadoop-related projects at Apache including the popular scalable machine learning and data mining library called Mahout, written also in Java. 

Hadoop builds on the MapReduce and GFS foundation, designed to abstract away much of the complexity of distributed processing running on large clusters of commodity computers. The MapReduce programming model allows developers to write parallel distributed programs very easily by only implementing two functions called \textit{Map} and \textit{Reduce}. Developers don't need to worry about doing a complicated code to e.g. distribute work to computers, internal communication, data transfers and dealing with fault tolerance. Instead they can focus on the logic to solve the problem at a hand.

\subsection{Programming Model}
As mentioned before a developer only needs to implement two functions Map and Reduce. The \textit{Map} function takes as input pair and produces an intermediate $(key,value)$ pair. The Map functions are run in parallel and produce many intermediate output pairs which is then grouped together with the same \textit{key} by the MapReduce framework and is passed along to the \textit{Reduce} function. The Reduce function receives a key and all of its set of values from all the Map functions, then it merges these values and typically produces zero or one output key and value pair per Reduce invocation. 

\textbf{Example} The counting occurrences of words in a document problem. The Map function receives each word as input and emits the word as a key with count of 1, e.g. we have set of words $w_1,...,w_n$ in a document then the output from all Map function would be the sequence of key value pairs: $(w_1,1),...,(w_n,1)$. The MapReduce framework then groups and merges all the $(key,value)$ pairs from all the Map functions and invokes the Reduce function with input key as a word and list of all the counts as the values: $(w_i,[v_1,...,v_n])$ where $w_i$ is a specific word with $1,...,n$ counts of 1, e.g. $(w_1, [1,1,1])$ if $w_1$ had 3 occurrences in the document. In the reduce function it will simply sum up all the counts and output a total count for each word. See Figures [TODO REFERENCES] below for examples how the Map and Reduce functions are implemented for the counting of words problem.


\textit{TODO Pseudo code of the Map and Reduce function}

The MapReduce also allows the developer to implement a \textit{Combiner} function to do a partial reducing task that is executed on the same computer node as the Map function. 


\subsection{MapReduce Hadoop Execution}

\subsection*{Map Task}

\subsection*{Reduce Task}

\subsection*{Combiner}


\textit{TODO Draw picture of the MapReduce framework}


\lipsum[7-8]


 % Background Theory 

% Chapter 3

\chapter{Related Work} % Write in your own chapter title
\label{Chapter3}
\lhead{Chapter 3. \emph{Related Work}} % Write in your own chapter title to set the page header

In subsequent sections we will give a overview of some of the related and recent work. We start with looking into work related to Clustering Player Behaviors and then we dive into numerous work related to clustering large set of data using the k-means algorithm, where we can have a finite stream of data, an endless and evolving data stream and finally some parallel clustering implementations in the Map-Reduce framework. In our work we focus on processing large amount of data of finite length in parallel using Map-Reduce that arrives each day. The behavior of the data can change from day to day like when dealing with an endless stream of data but is not in the scope and is discussed in future work in the Conclusions section. [TODO].

\section{Clustering Player Behaviors}
Many researches have been done on clustering and predicting player behaviors over the years to get a better understanding which kind of groups of behaviors are playing a game by exploring user telemetry and player events that is being logged as a the game is being played.

\section{Clustering Large Data}
K-means is one of the most studied clustering algorithm out there and is still actively researched. It's a simple algorithm that partition the data into \textit{k} partitions by minimizing its objective function sum of squared error. From its appearance in a standard version by Stuart Lloyd in 1982 \citep{Lloyd:1982} , it has been one of the most popular clustering algorithm to research because of its simplicity. There are many different research areas regarding k-means. The problem with manually set the number k of partitions, initializing the centers for the partitions and dealing with outliers in data. In recent years k-means has also been very popular algorithm to study in the Map-Reduce framework where the k-means algorithm can easily be applied to cluster large amount of data sets in parallel. We start with going through work relates to both working with a stream of data of length \textit{n} that doesn't fit in memory and the incrementally evolving characteristics of clustering endless data streams.

We start with a famous work by Guha et al. \citep{Guha:2003} where they introduce the STREAM algorithm that is based on the divide-and-conquer strategy and achieves a constant-factor approximation solving the k-median problem (a k-means variant). The algorithm divides the dataset into m pieces of similar sizes. Each of the pieces are independently clustered sequentially and all the centers from all the pieces are then clustered further. They show a new k-median algorithm called LSEARCH that is used by the stream algorithm and is based on local search algorithm solving the facility location problem \citep{Charikar:1999} to solve the k-median problem. Results show that LSEARCH produced better quality clusters than k-means and the hierarchical algorithm BIRCH \citep{Zhang:1996} but took longer to run.

In 2009 Ailon et al. \citep{Ailon:2009} extended the the work of Guha et al. mentioned above and showed an one pass streaming algorithm for k-means with approximation guarantees. Achieving that they introduced a new algorithm called k-means\# that builds on the non-streaming algorithm k-means++ by Arthur et al. [CITE] that is a combined algorithm of seeding the initial centers and running k-means. k-means\# provides a bi-criterion $ (\alpha,\beta) $ approximation algorithm by choosing $\alpha * k $ centers with approximation factor $\beta$. In the divide-and-conquer strategy they run the k-means\# independently on each piece of the data to achieve $O(k log k)$ random centers non-uniformly and use the k-means++ algorithm introduced in work of Arthur et al. [CITE] to find k centers from the intermediate centers found in all of the pieces of the data set. 

Another approach using a coreset by selecting a weighted subset from the original data set such that by running any k-means algorithm on the subset will give near similar results to running k-means on the original data set. Ackermann et al. [CITE] showed a new algorithm called StreamKM++ that uses k-means++ algorithm from Arthur et al. to solve k-means on the subset and also introduce a new data structure called coreset tree to speed up the time for the sampling in the center initialization. StreamKM++ is a streaming version of k-means++ to cluster large data sets. Their approach was shown to be on par with LSEARCH algorithm in cluster quality but outperformed BIRCH by factor of 2. A recent work by Shindler et al. [CITE] proposed an algorithm called \textit{Fast streaming k-means} based on the online facility location algorithm [CITE Adam Meyerson] and extends the work of TODO [CITE] by proving faster running time and a better approximation factor. Their algorithm outperformed both work of Ailon et al. [CITE] and Ackermann et al. [CITE].  

Adam Meyerson 
Online facility location. italic{In FOCS}, 2001


A different approach was introduced by Sculley in 2010 [CITE]. Using sampling and a gradient descent, Sculley [CITE] introduced a new algorithm called Mini-batch k-means that yields excellent clustering results with low computational cost for large datasets. The algorithm initializes the k centers like the normal k-means algorithm but then for each iteration it picks $b$ examples from the dataset and for each point in the example the center which it is closest to is updated by taking a gradient descent step towards the point, with a learning rate of that center. The approach scales when datasets grow large with redundant examples and allowing convergence to better solutions compared to the on-line stochastic gradient descent (SGD) variant proposed by Bottou et al. [CITE].

Turning to clustering a data stream where data arrives continuously and the behavior of the stream can change over time. This area of research is very popular in the recent years where generating data is rapid, has an unknown length and not possible to access historic data points seen before because of the amount of data. 

MapReduce.
A parallel version of k-means++ initialization algorithm can be found in the work of Bahmani et al from 2012 [CITE]. The algorithm is called k-means$||$ and is implemented in the Map-Reduce framework and they show it outperforms the k-means++ algorithm in both sequential and parallel settings.




\subsubsection{Map-Reduce}

\subsection{Data Streams}

 % Related Work

% Chapter 4

\chapter{Methodology} % Write in your own chapter title
\label{Chapter4}
\lhead{Chapter 4. \emph{Methodology}} % Write in your own chapter title to set the page header

\textit{TODO Describe introduction to the methodology}

\lipsum[1]

\section{Data and Preprocessing}

\textit{TODO Describe the real game dataset and preprocessing}

\lipsum[1-3]

\subsection{Feature selection and behavioral variables}
\textit{TODO Describe feature selection and behavioral variables}

\lipsum[1-3]


\section{K-means algorithm in MapReduce}
\textit{TODO Describe the k-means algorithm implementation in MapReduce and the relevant pseudo codes and pictures}

\lipsum[1-2]

\subsection{Map}
\textit{TODO Describe the map function}

\lipsum[1-2]

\subsection{Combine}
\textit{TODO Describe the Combine function}

\lipsum[5-6]

\subsection{Reduce}
\textit{TODO Describe the Reduce function}

\lipsum[7-8]

\subsection{Distance measure}
\textit{TODO Describe the distance measure used in k-means}

\lipsum[1]


\section{Experiment Set-Up}
\textit{TODO Describe the set-up for the experiments}

\lipsum[1-3] % Methodology

% Chapter 5

\chapter{Results} % Write in your own chapter title
\label{Chapter5}
\lhead{Chapter 5. \emph{Results}} % Write in your own chapter title to set the page header

In this chapter the results from the conducted experiments from Section~\ref{sec:experimentsetup} are presented.

\section{Cluster Quality - Real Dataset}
\label{sec:cq_realdataset}
Here are the result after incrementally cluster the ten multiple data batches of real game data and measuring the SSE error after performing $n$-iterations on each data batch. The SSE error gives an idea of the cluster quality at each data batch when using the k-means centroids results from previous data batch as an input.

Three different variations of $n$ were compared to see if there is any difference in doing a few iterations versus many iterations over each data batch. The main idea is to have a method that has a stable SSE error over multiple batches of data and have the desirable feature of having a downward trend for the SSE. 

Ten test runs were performed with different initial centroids from the first data batch trying to have the centroids to start at different points in space, creating different clusters and centroids as output from the first batch which can affect the consecutively batches. 

As we can see from Figure~\ref{fig:AvgSSEEachDataBatch} all the methods have a downward trend but are increasing the error with different amounts in batch $7-10$, tho the \textit{Iteration~1} method is showing a more stable change towards the final batch.


\begin{figure}[ht]
\centering
\includegraphics[trim = 10mm 70mm 10mm 80mm, clip, width=0.75\textwidth]{Figures/experiments/zdataWO_AvgSSEEachDataBatch.pdf}
\caption{This figure compares the average SSE error after clustering each data batch, using the three different $n$-iterations methods. }
\label{fig:AvgSSEEachDataBatch}
\end{figure}

After incrementally clustering the ten data batches with the different $n$-iterations methods, they all represent very similar clusters in the last batch, see Table~\ref{tab:results_avgSSEeachDataBatch}. The \textit{Iteration~2} and \textit{Iteration~5} has only about $0.6\%$ difference even tho the \textit{Iteration 5} is constantly showing lower SSE error in the first $7$ batches but then takes a steeper SSE increase up to the last batch. In this case the \textit{Iteration~5} method was too aggressive by trying to fit the centroids from the previous batch to the current batch by doing $5$-iterations.

\begin{table}[h]
\centering
\begin{tabular}{| l | r | r |}
    \hline
    & \textit{Iteration 2} & \textit{Iteration 5} \\ \hline
    \textit{Iteration 1} & $2.60\%$ & $3.19\%$  \\ \hline
    \textit{Iteration 2} & & $0.60\%$ \\ \hline
\end{tabular}
\caption{This table shows the average SSE error difference on the last data batch, between the three methods.}
\label{tab:results_avgSSEeachDataBatch}
\end{table}


In Figure~\ref{fig:results_AvgIncreaseEachTestRun} we can see on average how much was a single increase between one set of data batches for all test runs. The average single increase for the \textit{Iteration~1} method in each test is very different than the other methods. The reason is that are different initial centroids for each test and the \textit{Iteration~1} method only performs one iteration each time when clustering a data batch, doing so the method takes only one step towards the current data batch, slowly changing the means. If a data batch introduces a much different distribution this method will give worse results than the others because they are taking more steps towards the new data, but risking to deviate too much from the general population seen so far.


\begin{figure}[ht]
\centering
\includegraphics[trim = 10mm 90mm 10mm 100mm, clip, width=0.75\textwidth]{Figures/experiments/zdataWO_AvgSingleSSEIncreaseBetweenDataBatchesEachTestRun.pdf}
\caption{Here we can see on average the single SSE error increase that was measured, when SSE increases from one batch to another, for all the test runs.}
\label{fig:results_AvgIncreaseEachTestRun}
\end{figure}


When looking at the Figure~\ref{fig:results_AvgIncreaseBetweenDataBatches} we can see the single SSE error increase on average over all the ten test runs. The \textit{Iteration~1} method has considerable lower single SSE error increase compared to the \textit{Iteration~2} and\textit{Iteration~5} methods. The latter two methods being very similar to each other, both moving the centroids more aggressively away from already seen centroids.


\begin{figure}[ht]
\centering
\includegraphics[trim = 10mm 90mm 10mm 100mm, clip, width=0.75\textwidth]{Figures/experiments/zdataWO_AvgSingleSSEIncreaseBetweenDataBatches.pdf}
\caption{Here we can see the average SSE error increase between two data batches, for the all the test runs. }
\label{fig:results_AvgIncreaseBetweenDataBatches}
\end{figure}

The results show that the \textit{Iteration 1} method has the lowest average single SSE error increase between a set of data batches. We now look at the aggregated SSE increase for each of the test runs, see ~Figure~\ref{fig:TotalSSEIncreaseEachTestRun}. 

\begin{figure}[ht]
\centering
\includegraphics[trim = 10mm 90mm 10mm 100mm, clip, width=0.75\textwidth]{Figures/experiments/zdataWO_TotalSSEIncreaseEachTestRun.pdf}
\caption{This figure shows the aggregated SSE error between two data batches for each test run.}
\label{fig:TotalSSEIncreaseEachTestRun}
\end{figure}

The \textit{Iteration~1} method shows the lowest aggregated SSE error increase in all test runs, except in test run $5$. The random initial centroids from the first batch poorly represents the main distribution in the data and by performing $1$-iteration takes a longer time to adjust the centroids towards the mean of population. In Figure~\ref{fig:AvgSSEIncrease} is the average aggregate SSE error shown, over all the ten runs peformed.


\begin{figure}[ht]
\centering
\includegraphics[trim = 10mm 90mm 10mm 100mm, clip, width=0.75\textwidth]{Figures/experiments/zdataWO_AvgSSEIncrease.pdf}
\caption{This figure shows the average aggregated SSE error between two data batches, for all test runs.}
\label{fig:AvgSSEIncrease}
\end{figure}

\newpage

\section{Cluster Quality - Generated Dataset}
In the second cluster quality experiment a generated dataset is used. This dataset is generated with three normal distributions that changes and move from each other in the course of ten data batches. The idea is to measure the SSE error when facing a rapid changing data, in that sense that the distributions are moving between chunks of data. 

Finding a method that can represent a stable SSE error and a downward trend is of high value, as explained in the Section~\ref{sec:cq_realdataset}. The Figure~\ref{fig:results_gen_AvgSSEEachDataBatch} shows the SSE error that was measured on average for all the ten test runs when clustering the ten data batches.

\begin{figure}[ht]
\centering
\includegraphics[trim = 10mm 70mm 10mm 80mm, clip, width=0.75\textwidth]{Figures/experiments/gen_AvgSSEEachDataBatch.pdf}
\caption{This figure compares the average SSE error after clustering each generated data batch, using the three different $n$-iterations methods. }
\label{fig:results_gen_AvgSSEEachDataBatch}
\end{figure}

We can see that both the \textit{Iteration 1} and \textit{Iterantion 2} methods show a downward trend for the SSE error but the \textit{Iteration 5} method is taking a relatively big step from data batch $5$ to $6$. Methods \textit{Iteration 2} and \textit{Iteration 3} show the lowest SSE error for the last data batch, in Table~\ref{tab:results_gen_avgSSEeachDataBatch}, with only $2.36\%$ difference, and there is large gap of $22.78\%$ between \textit{Iteration 1} and \textit{Iteration 5}.

\begin{table}[h]
\centering
\begin{tabular}{| l | r | r |}
    \hline
    & \textit{Iteration 2} & \textit{Iteration 5} \\ \hline
    \textit{Iteration 1} & $20.91\%$ & $22.78\%$  \\ \hline
    \textit{Iteration 2} & & $2.36\%$ \\ \hline
\end{tabular}
\caption{This table shows the average SSE error difference on the last generated data batch, between the three methods.}
\label{tab:results_gen_avgSSEeachDataBatch}
\end{table}

This large gap between \textit{Iteration 1} and the other methods is because in this dataset the three distributions are changing relatively fast, moving it in a determined direction in each data batch. Making it hard to keep up with the changes with $1$-iteration while the other methods are moving closer to each next data batch.

In Figure~\ref{fig:results_gen_AvgIncreaseEachTestRun} we can see the measured SSE error when it increases between data batches, e.g. where a SSE error for a batch is higher than the previous data batch. The \textit{Iteration 2} method is now showing the lowest error on average, and in $4$ out of $10$ tests it almost doesn't increase the SSE. Method \textit{Iteration 5} is however very interesting in test $3$ where it much lower than the other methods. 

\begin{figure}[ht]
\centering
\includegraphics[trim = 10mm 90mm 10mm 100mm, clip, width=0.75\textwidth]{Figures/experiments/gen_AvgSingleSSEIncreaseBetweenDataBatchesEachTestRun.pdf}
\caption{Here we can see on average the single SSE error increase that was measured, when SSE increases from one batch to another. }
\label{fig:results_gen_AvgIncreaseEachTestRun}
\end{figure}

Looking into the data it shows that the \textit{Iteration 5} method manages to report much lower SSE error than the other methods when processing data batch number 6, using the centroids found in data batch number 5. And in general all the methods had a SSE increase when processing batch 6, see Figure~\ref{fig:results_gen_DifficultDataBatches} which data batches gave problems over all the tests. Batch 6, 9 and 10 gave high SSE increase on average, but the batch 6 had the far most error for all the methods. 

\begin{figure}[ht]
\centering
\includegraphics[trim = 10mm 90mm 10mm 95mm, clip, width=0.75\textwidth]{Figures/experiments/gen_DifficultDataBatches.pdf}
\caption{This figure shows the data batches where all methods increased the SSE error going from one data batch to the next. }
\label{fig:results_gen_DifficultDataBatches}
\end{figure}

When looking at the data population for data batch 5 and 6 we can see that there is a clear change in the data, going from batch 5 to 6, see Figure~\ref{fig:results_gensubfigures}. The figure on to the right shows clearly that one of the clusters have disconnected from the rest, and this seems to give problems for all methods. \textit{Iteration 5} had the most troubles with batch 6 in general, most likely by moving its centroids to aggressively through many iterations and end up in a local optimum.



\begin{figure}[ht!]
        \begin{center}
        \subfigure{%
            \label{fig:first}
            \includegraphics[width=0.5\linewidth]{Figures/gmeans/4_genshiftmeans.png}
        }%
        \subfigure{%
           \label{fig:second}
           \includegraphics[width=0.5\linewidth]{Figures/gmeans/5_genshiftmeans.png}
        } %  ------- End of the first row ----------------------%
        \end{center}
    \caption{These figures shows how the data distribution changes from data batch 5 to 6. In the right figure we can see a cluster that has moved from the other two clusters.}
   \label{fig:results_gensubfigures}
\end{figure}

For the \textit{Iteration 5} method from test 3 in Figure~\ref{fig:results_gen_AvgIncreaseEachTestRun}, it is likely that the random initial centroids for the first batch, leads the centroids very close to the highest population such that it didn't fell in to a trap when processing batch number 6, thus measuring low SSE for that method. The same happens in general for the other methods since they move much slower and are closer to the main population.

Looking at the measured SSE error that happened between all sets of data batches, we can see in Figure~\ref{fig:results_gen_TotalSSEIncreaseEachTestRun} that \textit{Iteration 2} is reporting over all tests very low SSE increase but is high in tests 2, 6 and 10, which were related to the data batch number 6, discussed above.

\begin{figure}[ht]
\centering
\includegraphics[trim = 10mm 90mm 10mm 100mm, clip, width=0.75\textwidth]{Figures/experiments/gen_TotalSSEIncreaseEachTestRun.pdf}
\caption{This figure shows the aggregated SSE error between two data batches for each test run.}
\label{fig:results_gen_TotalSSEIncreaseEachTestRun}
\end{figure}

When aggregating all the SSE error between all the sets of data batches over all runs we get \textit{Iteration 2} with the lowest average, with \textit{Iteration 1} not far away.

\begin{figure}[ht]
\centering
\includegraphics[trim = 10mm 90mm 10mm 100mm, clip, width=0.75\textwidth]{Figures/experiments/gen_AvgSSEIncrease.pdf}
\caption{This figure shows the average aggregated SSE error between two data batches, for all test runs. }
\label{fig:results_gen_AvgSSEIncrease}
\end{figure}

\newpage
\section{General Player Behaviour}
The results from real game data cluster quality experiment in Section~\ref{sec:cq_realdataset} were used and interpreted to find the general player behaviour. The centroids from the last data batch when using the \textit{Iteration 1} method was used as basis for this experiment, from the test run that had the the lowest SSE. 

The three centroids from the final data batch, found by the MR k-means algorithm, describes the average behaviour of the three clusters found, after clustering incrementally the ten data batches. See Table~\ref{tab:results_centroids} for the centroids and their z-score values, and Figure~\ref{fig:results_zscoredataset} for the visualization of the clusters found.

\begin{table}[h]
\centering
\begin{tabular}{| l | r | r | r |}
    \hline
    & \textit{Logins} & \textit{Battles} & \textit{Premium spent} \\ \hline
    Centroid 1 & $2.25$ & $1.35$ & $0.06$  \\ \hline
    Centroid 2 & $-0.29$ & $-0.33$ & $-0.31$  \\ \hline
    Centroid 3 & $0.24$ & $1.46$ & $2.62$ \\ \hline
\end{tabular}
\caption{This tables shows the how the final centroids in \textit{z-score} values looked like after clustering the final game data batch file, using the \textit{Iteration 1} method.}
\label{tab:results_centroids}
\end{table}


\begin{figure}[here]
\centerline{\includegraphics[trim = 10mm 10mm 10mm 20mm, clip, width=0.75\textwidth]{Figures/experiments/zscoreResultsGamedatasetwooutliers.png}}
\caption{This figures shows how the clusters looked like for the final data batch, the centroids are marked with stars.}
\label{fig:results_zscoredataset}
\end{figure}


These standardized values were transformed to their original values as explained in the experiment set-up in Section~\ref{sec:experiment_gpb}. For the original values see Table~\ref{tab:results_centroidsoriginalvalues}. The centroids represent the general player behaviour based on the behavioural features that were used, and can be used to describe behavioural profiles that is found in the data. See Table~\ref{tab:results_playerprofiles} for the player profiles that were extracted and interpreted from the centroids. The profile titles and their descriptions were made from a guess and intuition.

\begin{table}[h]
\centering
\begin{tabular}{| l | r | r | r |}
    \hline
    & \textit{Logins} & \textit{Battles} & \textit{Premium spent} \\ \hline
    Centroid 1 & $3.80$ & $2.86$ & $1.51$  \\ \hline
    Centroid 2 & $1.14$ & $1.10$ & $1.12$  \\ \hline
    Centroid 3 & $1.70$ & $2.97$ & $4.19$ \\ \hline
\end{tabular}
\caption{This tables shows the centroids in the original range of values for each game metric, used for interpretability.}
\label{tab:results_centroidsoriginalvalues}
\end{table}

\begin{table}[h]
\centering
\begin{tabular}{| l | r | l | l |}
    \hline
    \textit{Profile title} & \textit{\% of P} & \textit{Description} \\ \hline
    Active Joe & $9.39\%$ & Explores areas and fights battles \\ \hline
    Lazy & $81.12\%$ &  Jumps in and out of the game \\ \hline
    Golden Warrior & $9.49\%$ & Fights battles and spends in-game money \\ \hline
\end{tabular}
\caption{This table shows the player profiles and their size in the player population. Extracted from the centroids that describe the $k$ average player feature vectors after incrementally clustering multiple batches of game data.}
\label{tab:results_playerprofiles}
\end{table}


\newpage
\section{Scalability}
The results from the scalability experiment shows that the MR k-means algorithm is scalable algorithm that is fully capable to be run on Amazon EMR web service, launching on-demand Amazon Elastic Hadoop clusters. In Figure~\ref{fig:results_scalevertical} we can see the vertical scaling, where the computing power is doubled but processing the same size of data. 

\begin{figure}[ht]
\centering
\includegraphics[trim = 10mm 90mm 10mm 90mm, clip, width=0.75\textwidth]{Figures/experiments/scale_vertical.pdf}
\caption{This figure shows the running time (in sec) when doing vertical scaling for the MR k-means algorithm. Doubling the computing power on same dataset size.}
\label{fig:results_scalevertical}
\end{figure}

The horizontal scaling results are in Figure~\ref{fig:results_scalehorizontal}, and show the running time when doubling the number of computers or nodes when clustering the same size of data.  


\begin{figure}[ht]
\centering
\includegraphics[trim = 10mm 90mm 10mm 90mm, clip, width=0.75\textwidth]{Figures/experiments/scale_horizontal.pdf}
\caption{This figure shows the running time (in sec) when doing a horizontal scaling for the MR k-means algorithm. The number of computer nodes are doubled each time on the same size of data. }
\label{fig:results_scalehorizontal}
\end{figure}

The MR k-means scales both horizontally and vertically, showing the biggest performance jumps when doubling in the beginning of the tests. In Table~\ref{tab:results_scale_scaling} we can see how the MR k-means scales when doubling both the number of computer nodes and dataset sizes, resulting in nearly the same execution time for all tests.

\begin{table}[h]
\centering
\begin{tabular}{| l | l | l |}
    \hline
    \textit{Size} & \textit{\# nodes} & \textit{time in sec} \\ \hline
    220 MB & 2 & 280  \\ \hline
    440 MB & 4 & 280  \\ \hline
    880 MB & 8 & 285 \\ \hline
    1760 MB & 16 & 285 \\ \hline
\end{tabular}
\caption{This table shows the execution time for the MR k-means algorithm when doubling the computer nodes and the dataset size.}
\label{tab:results_scale_scaling}
\end{table}


\newpage
\section{MR K-means Combiner vs. non-Combiner}
Here we look at the running time results the final MR k-means algorithm was compared with it self but with the Combiner function removed from the MapReduce execution flow. The results show that the Combiner is an essential feature when there is a possibility to reduce problems in the Mapper phase, thus minimizing the data traffic and allowing more efficiency in the MapReduce framework, see Figure

\begin{figure}[ht]
\centering
\includegraphics[trim = 10mm 90mm 10mm 90mm, clip, width=0.75\textwidth]{Figures/experiments/combiner.pdf}
\caption{This figure compares the MR k-means algorithm with and without the Combiner function in MR implemented. We can see that having a Combiner can quickly become very efficient versus the non-Combiner version. }
\label{fig:results_combiner}
\end{figure}

\newpage
\section{Nearest Centroid Computation Methods}
The execution time for the three nearest centroid computation methods were compared. The Mapper phase in MapReduce is responsible to find the nearest centroid for each data point, and this computation is the most expensive work when running k-means.

The running time for the naive and the vectorisation versions were compared, leaving the distance matrix implementation out of the picture, for now. See Figure~\ref{fig:results_nearestcentroid} for the comparison of the two. On the figure we can see that the vectorisation version is performing much faster but it converges when running on 8 computer nodes, the MapReduce framework overhead takes over and to make the algorithm run faster we can scale-up by upgrading the computing power.

\begin{figure}[ht]
\centering
\includegraphics[trim = 10mm 70mm 10mm 70mm, clip, width=0.75\textwidth]{Figures/experiments/nearestcentroid.pdf}
\caption{This figure compares the execution time for the naive and vectorisation methods when computing the nearest centroid.  }
\label{fig:results_nearestcentroid}
\end{figure}

Running the distance matrix version to compute the nearest centroid doesn't work so well on Amazon EMR, compared to the other methods, see Figure~\ref{fig:results_nearestcentroiddist} for the running time for the distance matrix version. Running with 2 nodes resulted in error in the MR execution process and other execution times are much higher, e.g. running with 8 nodes in this setup the distance matrix is $\approx 10$ times slower.

\begin{figure}[ht]
\centering
\includegraphics[trim = 10mm 90mm 10mm 90mm, clip, width=0.75\textwidth]{Figures/experiments/nearestcentroid_dist.pdf}
\caption{This figures shows the execution time when using the distance matrix method, when computing the nearest centroid.}
\label{fig:results_nearestcentroiddist}
\end{figure}

 % Results and Discussion

% Chapter 6

\chapter{Conclusions} % Write in your own chapter title
\label{Chapter6}
\lhead{Chapter 6. \emph{Conclusions}} % Write in your own chapter title to set the page header


\section{Conclusions}
\textit{TODO Write Conclusions. Convince the reader that the research question was answered/solved. Write what is relevant to the research question. Use short statements directly related to the research question. }

\lipsum[1-2]

\section{Summary of Contributions}
\textit{TODO Describe the contributions may overlap Conclusions (Note: maybe move into Conclusions section). Short numbered statements.}

\lipsum[1]


\section{Future Research}
\textit{TODO What is the future work. What can be done differently, what needs to be addressed?}

\lipsum[1-2] % Conclusions

%% Chapter 7

\chapter{Conclusions} % Write in your own chapter title
\label{Chapter7}
\lhead{Chapter 7. \emph{Conclusions}} % Write in your own chapter title to set the page header

Data is all around us and is inevitable, it has been around for a quite some time but is getting bolder and more complex to understand and work with. Using services like Amazon Elastic MapReduce web service and their simple web storage (S3) is a right step analyzing and understanding these gigantic volumes of data. Implementing algorithms that can be distributed and executed in parallel is a very powerful tool when using it on large-scale data. 

Free online games (Free-to-Play) are increasing each day through, e.g. Facebook and Google Plus, generating revenue through in-game transactions. Data analysis and better understanding the customers and their player behaviour is a vital information to keep the customer in the game and to buy more virtual items using micro-transactions with real money. The games can be very complex and offer wide range of events and behaviour where millions of users interact to each other.

The goals for this thesis were achieved and the contribution are the following:
\begin{itemize}
	\item A MapReduce k-means algorithm that can find general player behaviour in the real game data set provided by GameAnalyltics, by incrementally cluster multiple batches. And were interpreted to player profiles.
	\item The algorithm finds stable and quality clusters while incrementally clustering multiple data batches, both for real game data and fictive generated datasets, where centroids move rapidly between batches.
	\item The algorithm uses an efficient MR implementation function called Combiner, to minimize network and data traffic inside the MapReduce framework.
	\item The algorithm uses efficient computation vectorisation approaches to perform fast operations on array like data objects, when computing the nearest centroid in the Mapper phase inside MR.
\end{itemize}



\subsection{Future Work}
\begin{itemize}
	\item Evaluate the algorithm over larger time periods, more number of data batches and in larger size and for different real life datasets. Interesting would to see if the algorithm continues to be stable.
	\item Evaluate which number of iterations would give the best results to certain types of evolving datasets.
	\item Looking into Iterative MapReduce frameworks that are much more efficient to deal with many iterations, by running a internal loop, meaning a separate control program is redundant.
	\item Evaluate if there is a possibility to use a distance matrix computation efficiently to find the nearest centroid running on Amazon EMR, without running into memory troubles. E.g. using more powerful \textit{Master Controller} instance, or higher memory instance running on Hadoop.
\end{itemize}
 % Conclusion

%% ----------------------------------------------------------------
% Now begin the Appendices, including them as separate files

\addtocontents{toc}{\vspace{2em}} % Add a gap in the Contents, for aesthetics

\appendix % Cue to tell LaTeX that the following 'chapters' are Appendices

\input{./Appendices/AppendixA}	% Appendix Title

%\input{./Appendices/AppendixB} % Appendix Title

%\input{./Appendices/AppendixC} % Appendix Title

\addtocontents{toc}{\vspace{2em}}  % Add a gap in the Contents, for aesthetics
\backmatter

%% ----------------------------------------------------------------
\newcommand*{\doi}[1]{\href{http://dx.doi.org/\detokenize{#1}}{doi: \detokenize{#1}}}
\label{Bibliography}
\lhead{\emph{Bibliography}}  % Change the left side page header to "Bibliography"
\bibliographystyle{unsrtnat}  % Use the "unsrtnat" BibTeX style for formatting the Bibliography
\bibliography{Bibliography}  % The references (bibliography) information are stored in the file named "Bibliography.bib"

\end{document}  % The End
%% ----------------------------------------------------------------