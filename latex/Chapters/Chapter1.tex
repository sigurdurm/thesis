% Chapter 1

\chapter{Introduction} % Write in your own chapter title
\label{Chapter1}
\lhead{Chapter 1. \emph{Introduction}} % Write in your own chapter title to set the page header

\section{Motivation}
We live in a world where data is being generated at an amazing rate everywhere around us. Data can describe characteristics of e.g. Internet activities, social interactions, user behaviors in games, mobile phone activities, scientific experiments and measurements from different devices and sensor equipments. Amount of data that is being registered and stored around us is growing massively in volume and complexity. Organizations are storing more and more historic data than before, moving from large databases towards more commonly online distributed file systems or storage web services providing scalability and high availability at commodity costs where \textit{petabytes} ($10^{15} = 1.000$ \textit{terabytes}) of information can be stored. We live in a world of \textit{Big Data}, exploring unknown patterns and structures without knowing where it will lead us in the future. 

\null
% Now comes the "Funny Quote", written in italics
\textit{``Information is the oil of the 21st century, and analytics is the combustion engine.''}

\begin{flushright}
Peter Sondergaard, senior VP at Gartner
\end{flushright}

In digital games, data about in-game user interactions have been logged and behaviors analyzed since the first game came out. Analyzing the user experience and behaviors of players have mostly been done in laboratories in the past, both during game development and after game launch to see if the game was played as designed. Game designs have becoming increasingly complex in the recent years offering much more freedom to the players by increasing the number of actions available, items to interact with and Massively Multi-player Online (MMO) persistent worlds that continue to exist after a player exits a game \citep{Kim:2008Tracking, Drachen:2011Evaluating}. This complexity generates much more user-centric data than before and is increasingly challenging when evaluating game designs \citep{Pagulayan:2002UserDesign, Seif:2013GameAnalytics}. The user interactions being registered is called \textit{user telemetry} and is translated to \textit{game metrics} as referred in game development, providing detailed and objective numbers, e.g. total playtime, monsters killed, puzzles solved.

Collecting user's telemetry can give very detailed quantitative information on player behavior and using data mining techniques can supplement traditional qualitative approaches with large-scale behavioral analysis \citep{Yannakakis:2012}, for example show where users are getting stuck and finding actionable behavioral profiles \citep{Kim:2008Tracking, Drachen:2012, Drachen:2011Evaluating}. In the recent years user behavior analysis have in part been driven by the emergence of MMO games and Free-to-Play (F2P) games which can have millions of users and objects that can form highly complex interactions. These game models, especially of persistent nature, are monitoring users actions and their behaviors to drive their revenue with subscriptions or offer players to buy virtual items via micro transactions \citep{Kim:2008Tracking, Drachen:2011Evaluating, Fields:2011SocialGame, Seif:2013GameAnalytics}. 

One way of doing a behavioral analysis is use an unsupervised machine learning technique called clustering. Cluster analysis is a popular exploratory data mining technique that groups set of data objects together in a cluster that are more similar to each other than data objects in other groups \citep{Xu:2005Clustering}. Human beings categorizes or classifies a new object or a phenomenon based on similarity or dissimilarity of the object's descriptive features and is one of most primitive activities of humans \citep{Anderberg:1973ClusterAnalysis}. Clustering explores the unknown patterns of the data and provide compressed data representation for large-scale data. In computer games cluster analysis or behavioral categorization can find behavioral profiles that are actionable and give high valuable insights into the game development as well as increasing the monetization \cite{Drachen:2009Tomb, Mahlmann:2010Tomb}. 

Many clustering algorithms are designed for modern sizes of datasets where the whole data can fit into memory or allowing few passes into a database (where each data object is read more than once). It can be very expensive analyzing large-scale datasets and to get answers efficiently then one needs to reduce the set of data to be analyzed, e.g. sample fewer players and have fewer features (dimensions) to be compared. Computations for large-scale data takes time and needs to be distributed to be able to complete in reasonable amount of time. Google's MapReduce programming model was introduced in 2004 \citep{Dean:2004} and allows automatic parallelization and distribution of computations on large clusters of commodity computers. Allowing programmers and researchers to easily implement highly scalable algorithms to process large amount of data using the MapReduce model without worrying about handling failures and distributing the data with a large amount of complex code. 

\section{Problem Statement}

{\addtolength{\leftskip}{5mm} How can clustering using incremental k-means find general player behaviors in large-scale behavioral game data in reasonable time? \par}

Considering the massive size of user telemetry data being logged and processed, and the complexity of game designs. There is a knowledge gap when it comes to analyzing such large-scale data efficiently. Number of players are increasing and the complexity of player-game and player-player interactions grows exponentially. The largest massively multiplayer online role-playing game (MMORPG) \textit{World of Warcraft}~\footnote{http://en.wikipedia.org/wiki/World\_of\_Warcraft} had a population around of 10 million users (in 2012 from MMOData~\footnote{http://mmodata.blogspot.com}), where players live in a persistent world that can create many millions of different and complex interactions in the game. 

User telemetry from games can arrive in daily chunks and need to be processed incrementally (in mini batches). There is a need for algorithms that can process massive amount of data that doesn't fit in a computer memory to extracts knowledge in a reasonable time. The k-means algorithm can find clusters that represent general game behaviors. When implemented in the MapReduce framework, k-means can cluster the data in parallel and is highly scalable with running time increasing linearly with the size of the input. 

Our goal with this project is to implement a scalable clustering algorithm to find the general behaviors in a specific real life game dataset in collaboration with GameAnalytics (GA) \citep{GA2013}. The goal is not to implement a complete product but a scalable algorithm that provides information about the general player behavioral profiles for a specific game. Also the algorithm should allow GA to easily develop a product that can cluster general player behaviors in large-scale games.

The success criteria of this project is:
\begin{itemize}
\item A scalable k-means clustering algorithm finding clusters describing the general behaviors of a real life game dataset provided by GA.
\item The general behaviors found must be intuitively interpretable and actionable to game developers.
\item The algorithm must be able to process incrementally cluster daily arriving chunks of game metric data.
\end{itemize}

GameAnalytics is Software as a Service (SaaS) start-up, a data and analytics engine for game studios with its headquarters located in Copenhagen. Analyzing large quantities of game metric data that needs to be processed efficiently returning actionable results to aid game design and development. 

\section{Method}
TODO
A short description of the methods

\lipsum[2]

\section{Contributions}
TODO
Description of the contributions made in the project

\lipsum[3]


\section{Project Outline}
References are cited by index in the bibliography and are in order of appearance, e.g. [2] is a citation number two that is referenced in the thesis. Referring to other sections is by the number of that section, e.g. 4.1.2.

The organization of the thesis is as follows: 
\begin{itemize}
\item \textbf{Chapter 2 - Background} Describes a short background theory about clustering player behaviors and the MapReduce framework for large-scale data parallel processing.
\item \textbf{Chapter 3 - Related Work} Overview of related and recent work regarding clustering player behaviors and large-scale data with k-means as focus.
\item \textbf{Chapter 4 - Methodology} Our design and implementation work is described; Description of the real game dataset and selection of features, the k-means algorithm in MapReduce and the experimental set-up.
\item \textbf{Chapter 5 - Results} Experiment results and observations are explained.
\item \textbf{Chapter 6 - Conclusions} Conclusions are drawn from the study including future research.
\end{itemize}
