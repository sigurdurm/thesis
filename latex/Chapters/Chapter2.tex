% Chapter 2

\chapter{Background Theory} % Write in your own chapter title
\label{Chapter2}
\lhead{Chapter 2. \emph{Background Theory}} % Write in your own chapter title to set the page header


\section{Player Behavior Profiles}
\textit{TODO Describe player behavior}

\lipsum[1-2]

\subsection{Game Metric}

\textit{TODO Describe User Telemetry and Game Metric. Features and behavioral variables}

\lipsum[2-3]

\section{Clustering}
\textit{TODO Describe clustering}

\lipsum[4-5]

\subsection{K-means Algorithm}
\textit{TODO Describe k-means}

Many clustering methods exists but one of the most popular ones is called the \textit{k-means} algorithm [Forgy CITE]. The k-means term was first used by MacQueen, J. (1967)[CITE] but the idea goes back to Steinhaus, H. (1957)[CITE]. K-means seeks to group data into \textit{k} partitions or clusters and gives insights into the general distribution in those clusters. The objective function in k-means minimizes the sum-of-squared-error. For example given set $S=\{1...,n\}$ set of $n$ data objects, where each data point $x$ is a real number $d$-dimensional vector $x_i \in \Re^d, i=1...,n$ and we want to partition them into \textit{k} clusters $C=\{C_1...,C_k\}$, then the objective function is defined as


\begin{center}
$J(C)=\displaystyle \sum_{k=1}^{K}\displaystyle \sum_{x_i \in C_k}\left \| x_i-\mu_k \right \|^2$ 
\end{center}

where $\left \| x_i-\mu_k \right \|^2$ is a chosen distance measure between a data point $x_i$ and its cluster center (mean) $\mu_k$, the function $J(C)$ is the overall distance of \textit{n} data points from their respective clusters.


TODO check below text and compare to history... Lloyd vs Forgy.

The standard algorithm was published by Lloyd, S. (1982) \citep{Lloyd:1982} but was first proposed by Lloyd, S. in 1957 inside Bell labs. The k-means algorithm also referred as the Lloyd's algorithm uses a an iterative refinement method. 




\lipsum[5]

\subsection{Player Behaviors}
\textit{TODO Describe clustering player behaviors with focus using k-means}

K-means algorithm have been shown to be very useful in behavioral analysis to give good insights in the general behaviors found in a game [CITE].

\lipsum[6]


\section{MapReduce and Large-Scale Data}
\textit{TODO Describe MapReduce and Large-Scale data}

\lipsum[7-9]


